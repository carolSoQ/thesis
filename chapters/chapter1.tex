\chapter{Introduction}

\section{Background}

Postural researches from the past three decades focused on different parts of the human body and identified different factors that can influence health. The findings of Bergqvist et al.~\cite{vdt_on_musculo_disorder}, Gerr et al.~\cite{epid_musculo_disorder}, and Dunne et al.~\cite{wearable_spinal_posture} all suggest a correlation between musculoskeletal discomfort and engagement in activities employing visual display terminal (VDT) devices with unhealthy postures. Among the research, four postures, including rotate the head by too large angles, sit with the lower back not being supported, maintain a same postures for long time, and cross legs, are repeatedly proved to be related to potential musculoskeletal disorders. The techniques developed to improve posture, such as the \textit{Lumo Back}~\cite{wearable_spinal_posture}, however, only targeted one or few bad postures and had very few feedback methods. The aim of the research is to constructing a posture improvement system that could detect multiple bad postures and provide effective feedback.

\section{Motivation}
Sometimes I will also experience musculoskeletal discomfort such as sour waist and tight shoulders. I contributed the symptoms to my improper sitting posture. Even though I understand that some postures could be harmful to my body, I find it very hard to remind myself to avoid those postures while focusing on other tasks. Therefore, I believe it would be useful to have a system that can detect the postures and provide appropriate feedback.

\section{System}
In the current research, the target postures were selected from a trade-off between findings from a posture guidance review and assumed relaxing user experience. Microsoft Kinect is used to obtain posture data based on the justifications from the findings in literature review. A feedback model was constructed to provide feedback to the detected posture using different feedback methods. An initial user study was conducted to gather the target posture data to build a posture classification model. In addition, the occurrence rate of different postures in two scenarios was also measured to set the baseline of the posture behaviour pattern. The user study also investigated issues related to the system design through an interview.

The system implements three different feedback methods. They are delivered through a projector and a mobile device. The general posture behaviour of all users in the environment is visualized, showing the impact of each individual's current posture. Users can also interactively explore the type and duration of their current posture.

The system evaluation was conducted in the second user study. The participants' posture behaviour were measured in the same scenarios as those in the prior user study in order to investigate the effects of the system on the occurrence rate for the target postures. In addition, an interview were carried out to examine to the usability of the system.

\section{Objectives}
The objectives of the research are to reduce the occurrence rate of the target postures in every scenario with a positive user experience of the system. To achieve this, the system should:

\begin{enumerate}
  \item Detect health behaviours sensitively and accurately
  \item Provide real time feedback in interesting way that would not irritate or bore the users
  \item Provide effective feedback that would make the users take action on improving the target postures
\end{enumerate}
