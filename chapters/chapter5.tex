\chapter{Feedback System}

\section{Overview}
Three functions chosen from the feedback model are implemented in this stage to construct the feedback system. The functions include \textit{Angel and Devil Clock}, \textit{Individual Head Tracking}, and \textit{Android Application}.

Angel and Devil Clock uses the projection system as the media to visualize how many users with either good and bad postures are currently detected in the environment. It also visualises the performance of the overall posture behaviour of all users in the environment, measured by the overall duration of their good posture and bad postures.

The second function is Individual Head Tracking. This function also uses the projection system as the media, and it will be triggered only it detects a user's bad posture. When the function is triggered, a spotlight will be projected onto the wall at the user's head position. If the bad posture had not been corrected in time, the spotlight would change into a blurry face, then a clear face, and eventually a body including a face and a skeleton. The joints affected by the current posture will be highlighted on the skeleton. The image will disappear immediately when the bad posture is corrected.

the bad posture reminding interval and good posture reinforcing interval

The android application is called \textit{ELF}, which is the abbreviation of \textit{Embedded Live-posture Feedback}. ELF demonstrates the feedback model when used cross-devices. The settings of the application can be personalised. The users can choose an avatar for their positive reinforcement, the reason to improve their posture behaviour, and the time interval to receive posture feedback. Once the application starts, it will serve in the background most of the time, and it will automatically jump out as a chat head when a good or bad posture of a user has remained over the user-defined interval. The user can click the chat head to open the messenger which shows the current posture type and its duration. The user can provide feedback by clicking one of the three emojis. The chat head then can be removed.

The system has two server-client connections. The computer connected to the Kinect serves as the first client, which passes the posture data to the first server. When the server receive the posture data, the compute would analyse the data and drive the projector to deliver appropriate Angel and Devil Clock and the Individual Head Tracking feedback. The same computer is also the second client that sends the posture analysis result to the server running on the mobile device.

\section{Angel and Devil Clock}

\subsection{objective}
The feedback aims to deliver the overall posture behaviour performance of all detected users in the environment. The rationale behind the design is hoping that social influence, such as social comparison, between users can impel them to improve their own posture.

\subsection{description}
All the graphics are hidden until any user in the environment is detected to have a good posture or bad posture. The angel figure will rise from the bottom at the first time when a user is detected as having a good posture. The same goes for the devil figure when a user is detected as having a bad posture. The two figures will be kept stationary until the system is closed. There will be some avatars on both the angel and devil’s arms. The number of avatars on the angel’s arm represents the number of the users who currently have good postures. On the other hand, the number of avatars on devil’s arm represents the number of the users currently having bad postures. The design of the avatars is a real-time reinforcement of good posture behaviours, because participants from the initial study wish that their good postures can be captured.

The angel is designed to give an impression that it is the better group, compared to the devil. It is shown through higher positioning of the angel compared to the devil, as well as happier avatars on the angel's arm compared to those those on the devil’s arm. According to the principle of social comparison, which states that people would reference others' behaviour and adjust their own behaviours based on their own judgements of what they observed~\cite{social_identity_comparison}, when some users are judged as belonging to the angel, other users would make extra effort to join the same party as the result of social comparison. If most users in the environment are judged as belonging to the angel’s party, there would be an effect of conformity on the remaining users. People tend to conform to what most people in the group do to avoid making themselves different from the norm~\cite{social_influence_complicance_conformity}.

The overall performance of all users' postures is visualized using tokens, where flowers represent good posture behaviours and boms represent bad posture behaviours. The performance score is calculated as the number of users with good postures multiply by the length of good postures minus the the number of users with bad postures multiply by the length of bad postures. If the performance score is positive, bombs will be removed and flowers will blossom. If the performance score is negative, flowers will gradually wither, and eventually when there are no blossomy flowers, bombs will appear one after another.

There is a good posture label and a bad posture label in the visualization. When a user initially joins either the angel or the devil, the corresponding label will appear for two seconds to indicate the user is represented by the newly added avatar on either the angel's or devil's arm. When a user changes from good to bad posture, or vice versa, the corresponding label will appear again. The labels are designed to remind the user's current posture behaviour, while reinforcing the association between the avatars and their posture.

\subsection{Design justification}
This feedback should give users a general understanding of their current posture behaviour. However, knowing their posture classification may induce the sense of fear or anxiety, for example, when users are classified into the devil's party. Therefore, the devil is drawn as a ridiculous, rather than terrifying, figure. In addition, the posture performance score is illustrated using flowers and bombs, as opposed to the original design of a clock. Not only do flowers and bombs depict the sense of positive and negative, respectively, feedback using static images compared to running clocks are less distracting to the users.

\section{Individual Head Tracking}
\subsection{Objective}
Individual Head Tracking occurs within the Angel and Devil Clock visualization. This feature emphasizes individual posture behaviours, whereas the Angel and Devil Clock focuses on both group posture patterns and the overall posture performance. This feedback has three levels, from subtle to obtrusive.

\subsection{Description}
When the system detects a user's bad posture, a spotlight will be projected onto the wall right in front of the user's head. The spotlight is a subtle feedback, as shown by its 30 percent transparency. The spotlight will also follow the user's head movement, until the user corrects their bad posture. The spatial position of the spotlight allows users in the environment to identify themselves, hence be aware of their bad postures.

Later, if the user has not corrected their posture after the spotlight is shown, a pair of eyes will be added to the spotlight, making the figure resemble a human face. According to Scheier \& Carver~\cite{self_focus}, the figure of eyes would facilitate the effect of self-focus, which makes the user more likely to be aware that they are currently exhibiting a bad posture.

Finally, when the user ignores the feedback again, a cartoonish skeleton will be displayed underneath the face. The body regions, or joints, that are affected by the bad posture will be highlighted. This feedback method is designed to provide users with information on the impact of bad postures on their body, which was rated the most useful by participants in the initial user study.

\subsection{Design justification}

The cartoonish skeleton is chosen over the default Kinect skeleton visualization in the SDK. There are three main reasons. Firstly, the Kinect has some uncertainties when it acquires joint positions, even when the user remains stationary. As a result, the user's projected skeleton joints would invariably move unpredictably along with the face. On the contrary, the cartoonish skeleton is a static image where all of the joints will move an equal amount of distance. Secondly, the posture reminder function is a supportive tool, hence the user should not pay too much attention on the details, for example, the joint positions. The user may spend more cognitive resources on the feedback visualization than necessary. Thirdly, users may find it embarrassing to share the skeleton joints with their companions. Moreover, this could threaten the privacy of self-space.

\section{ELF}
\subsection{Objective}
ELF provides posture feedback to users who are using their mobile devices and are likely to miss the information projected onto the wall. It also allows users to explore their current posture interactively.

\subsection{Description}
ELF is an android application. The homepage has two buttons - ``start'' and ``setting''. Users can click ``start'' to initiate the posture feedback service. Users can click ``settings'' to adjust the following settings: they can choose the preferred posture reminder interval, personal motivation for improving their posture, and the character which will praise their good posture.

The posture feedback service shows a small chat head of the elf saying ``I will serve you around.'' The chat head can be dragged to anywhere on the screen and hidden by dragging it to the bottom, just like the Facebook messenger chat head. If a bad posture has lasted for predefined, or the default, interval, the mobile device will vibrated. The chathead will reappear if it were hidden when a new feedback is received. The chat head image will have red edges if the user has bad posture or become an elf in Hawaii-vacation style if the user has good posture. If the user has a bad posture, the chat head will not be draggable until the user clicks on the chat head to open the messenger.

The messenger has three sections. The elf will display the current feedback with an animation, a short description, the duration of the posture, and a timestamp. The user can respond to the posture by clicking one of the three emojis to show their current feeling. The emojis include a determined face, a calm face, and an annoyed face. The system will display a text for each emoji the user choses. For example, if the calm face were chosen, the system would show ``Ok! I understand''. The emojis are an application of cognitive dissonance. The system displays a message that is far more positive than the actual emoji, even for the annoyed face. Therefore, the user would tend to adjust their actual feelings to keep it consistent with the displayed message. Then, the elf will reply to the said message, while reinforcing the same idea with cognitive dissonance.

When the user has a good posture, the system will display the character selected by the user in the setting panel saying ``Nice Posture!'', instead of an animation. According to the transportation theory proposed by Green (2004), when people like one character, they tend to identify with what the character says and does. The statement ``Nice Posture'' could be an effective reinforcement of the user's good posture behaviour.

\subsection{Design justification}
The elf in the home page wears a suit, which gives a sense of professionalism. According to the elaboration likelihood model proposed by Petty and Cacioppo~\cite{elaboration_modeL_persuasion}, the user would more likely be persuaded by information provided by an expert. The user study also shows that users tend to take the cognitive peripheral route when deciding their posture, that is, they do not think too much before doing it. The elf is displayed as a credible source which the users can use it to guide their posture behaviour. In addition, Petty and Cacioppo~\cite{elaboration_modeL_persuasion} also suggest that keeping people in a good mood would facilitate persuasion procedure when people take a peripheral route. This gives rise to the relaxing style of the graphics in the current application.

\section{Implementation}
These feedback contain many graphical materials. All of the graphics are designed and created originally, except the pictures of the characters used to complement good postures and the cartoonish skeleton.

The Angel and Devil Clock and the Individual Head Tracking are implemented using \Csharp{} in Visual Studio. The project is based on the Microsoft RoomaliveToolkit~\cite{roomalive_toolkit}, as it supports projection mapping for the system. Using the Roomalive Toolkit also enables other functions in the feedback model having the 3-dimension effect to be integrated in to current system. The drawing of the graphics feedback to be projected is conducted using the Windows Forms class. The computer connected to the projector is the server that receives the data passed from the computer connected to the Kinect, and it would use the classification model to render the appropriate feedback for current posture and drive the projector to deliver it. In the meanwhile, the computer would pass the data related to current posture to the android device to activate the ELF.

The ELF is implemented using Java in Android Studio. It includes four activities and six layouts. The mobile device is the server, which receive the data passed from the client. The data is used to determine the content of the feedback. The server and client are from Kinect2Kit\cite{kinect2kit}. The two server-client pairs integrate the functions of the posture detection, posture classification, and the three functions provided by the two devices for the system.
