\chapter{Conclusion and Future Work}

\section{Current Work}
In this research, the researcher built a posture feedback system with three feedback methods and decreased the occurrence rate of every target bad posture in every scenario.

The system defined its target bad postures for the system based on a trade-off between the combinations of health posture principles and a predictable comfortable and relaxing user experience, generated from the NHS guideline~\cite{nhs_sit_correctly}, Torres~\cite{best_viewing_distance_tv}, and Hargrove~\cite{bad_posture_back_pain} (See Table~\ref{tab:posture_descriptions}). A feedback model listing 33 candidate feedback methods (see Figure~\ref{fig:feedback_model_2}) is built to discuss the appropriate response to the detected posture.One of the main features of the feedback model is that it contains multiple output devices, which enables the researcher to broaden the diversity of candidate feedback methods. The feedback model also groups the candidate feedback methods by three intervention extends, which are subtle, subtle to obtrusive, and obtrusive respectively. The methods with less intervention extend would be delivered earlier to minimise the interferences to the user's’ current activities.

A posture classification model developed in the user study is responsible for the real-time posture type judgements.The model has high accuracy rate on classifying detected postures according to the result of validation and evaluation. Once a classification is made, an appropriate feedback would be delivered accordingly. The design of the feedback system are mainly based on the findings from the user study and the literature review. There are three feedback implemented in the system: Angel and Devil Clock, Individual Head Tracking, and the android application ELF. The functions are integrated using two pairs of client and server. The functions support the reinforcement for good postures, the illustration for overall posture behaviour performance, the active explore for the current posture, and the demonstration for the impact of current posture. The system also provides the cross-device feedback, using the projector and the mobile device in the same time to decrease the possibility for the user to miss the information.

The evaluation of the system presents the improvement of the bad posture occurrence rate on every scenario, and also shows an average rating 4.4 from the 5-point Likert scale for the eight dimensions (see table) of the system.

\section{Limitation}
There are four main limitations for the system. Firstly, the effect of the feedback for the stationary posture and the short viewing distance are not tested in the study. It is because the nature of a stationary posture, which is defined as all joints in a body region (with +-0.04 cm tolerance for the measured value) being kept in the same positions for 15 minutes in this research. The observation duration for the scenario is also 15 minutes only, as the scenario observation is just a part of the evaluation; if the overall length for the evaluation lasted too long, there could be a problem for the participant recruitment. On the other hand, if the definition of the stationary were modified for the purpose of testing the effect of the feedback for it, a user who keeps a good posture for a reasonable time would receive a bad posture feedback saying their good posture is too stationary, which would be disruptive and influence the trust for the user to the system. Therefore, the evaluation for the effect of feedback for the stationary posture is not available currently. A natural short viewing distance posture behaviour is also unlikely to be observed due to the limit of the lab setting.

The second is about the limitation of the Kinect, which is used as the only tool to gather the posture data. If a user keeps stationary for a while, the Kinect might regard the user as the background until the user moves again. This circumstance did not show during the tests, but it is likely to happen when the system is applied into real life. An additional condition could be added into the judgment system, such as, if the head position of any user were not cross the border of the Kinect’s observation scope before the user disappears, the user should be regarded as staying in the position where they are lastly detected, and the feedback for a stationary posture should be delivered.

The third is the predefined setting for the environment due to the use of a projector. Most projectors can only project clear image in a dark environment due to its principle of operation. Currently the projectors that can perform well in the bright environment are usually expensive and mainly for research use. However, the breakthrough of the technique and the popularization of the powerful projector are predictable.

The fourth limitation could be the biggest challenge for the system to be applied in real life, which is the reliability of current posture classification model. The current model was built using the data from only 10 participants and validated with the data of only 2 participants, with body height ranged between 156 and 172 cm. The amount of the participants is not representative to all potential users. A model built and validated with the data from more participants that cover all body height ranges would be more reliable and effective, and the demographical issues could also be considered into the procedure of the model construction. Building different models for different groups of the users could be a possible solution to achieve a good reliability and validity for the model. Also, the Kinect sensor. should be placed on the ceiling to prevent the blocking in real life application.

In addition, the user study for the watching TV scenario only measured the posture behaviour pattern of the participants who engaged in the activity alone, but the posture behaviour pattern of the participants engaged in this activity with a companion were also assessed in the evaluation. The comparison between the two scenarios could not really reflect the effect of the system for improving the posture behaviours of two users at the same time, as the baseline data is not of the same condition. The design of the experiment could be improved in the future research and a more reliable result could be found. Further, the scenarios of watching TV are not limited in individually or two persons, the scenarios for more people watching TV could be studied in the future research.

\section{Discussion}
\subsection{Social influence}
The system achieve its predicted effect in general, however, the improvement for the posture behaviour when two users watching TV together is not as good as the other scenarios, if the bias for using the data of watching TV with only one participant as the baseline were temporarily put aside. The expected power of social influence, such as social comparison and conformity, seemed not work significantly. This might because that the users did not notice the other people’s posture classification or care this enough, as this issue might not that matter to them to spend the cognitive resource on.

There would be two possible ways to enhance the effect of the social influence. The first is to increase the user's’ emphasis on the posture behaviour by providing more prior instructions about the importance of having good posture. If the user emphasises the posture behaviour more, they would be more likely to observe what posture the others have and what’s the classification for it. The effect of social comparison is therefore more likely to work. The other method is to use the power of the social cohesion, which is defined as the power that make people from the same society to cooperate with each other for the good of the society~\cite{elaboration_modeL_persuasion}. As the system is used for home setting, the users from a same environment would usually live together and be close to each other, which could be regarded as a team; if there is an online platform built for the performance of the team’s overall posture behaviour to be uploaded, and some appropriate and appealing reward could be given for the teams with good performance, having good posture behaviour could become a common goal of the users from a same team. The effect of social cohesion could happen.

\subsection{System performance}
The result from the evaluation could be an underestimate to the system's actual performance, since most of the participants presented they felt the effect of the system is novel and therefore intended to have many different postures to explore the system. The fact again backs the efficiency for the system in short term. On the other hand, the long-term performance for the system did not be evaluated. If the users get used to the stimuli provided by the system, the desensitization of the user to the system is possible to happen. The users would be more and more easy to ignore the feedback for their posture, especially for who hold a passive attitude for the posture improvement.

There could be two possible solutions for the desensitization issue. The first one is to modify and update the system from time to time. For example, change the pattern of the angel or the devil’s clothes, or change the tokens for good posture performance with different species of the flowers. However, if the change is not significant and interesting enough, the desensitization would still happen. The only way to ensure the system's long-term performance is to enhance the user’s emphasis on their posture behaviour. If the user put enough emphasis on the posture behaviour, they could become an active user and increase the priority of their limited cognitive resource use on the information provided by the system. Some diagrams or pictures, which indicates the importance of having good posture and are relevant to the issues that the user cares, can be provided every time the user starts the system to influence the attitude for the importance of good posture. For example, the diagram showing the difference of average thickness of legs between people who often have a cross-legs posture and people who don’t could be persuasive for the users who care their body shape very much.

\section{Contribution}
There are three main contributions of the research. The first one is the construction of an effective posture classification model. The model is linear and enables the classification to be made in real-time. Also, the performance of the model is good according to the participants’ rating (sensitivity = 4.1 and correctness = 4.0 on a Likert 5-point scale). Using the posture data from more participants in a future user study to modify the parameters in the formulas of current model could further increase its reliability.

Secondly, the feedback model generated in chapter 3 provides abundant candidate feedback methods. There are 34 feedback methods in the model and can be categorised by targeting posture, output device, and intervention extend. Future researchers who aim at designing the feedback to improve posture or regulate a specific can reference the model and modify the methods within to apply to their work.

The third and the most important contribution is the development of a new feedback system for posture behaviour. The system is very different from the past works sharing the same objective, such as a wearable device for slouch feedback~\cite{wearable_spinal_posture} and the ergonomic-based sitting apparatus for maintaining a good posture~\cite{monitor_support_apparatus}. It provides cross-device visual feedback and is able to recognise different type of bad postures. More importantly, the system emphases the user experience very much. The feedback it delivers would have only subtle intervention extend, which gradually increase along the duration of the detected bad posture. Also, the contents of the feedback are carefully chosen and what might cause anxiety or fear to the users is avoided.

\section{Future Work}
\subsection{Posture classification model}
A more reliable model is needed for the posture classification. It should be built using the data of participants that cover all body height ranges, and the demographical issues could also be considered. It might require multiple models that suit different groups of the users to ensure an accurate and reliable classification quality. In addition, currently the Kinect sensor is placed on the table in front of the users; however, the vision of the sensor would be occluded by other stuffs in the real life setting. The solution would be to install the Kinect sensor at the ceiling and observe the users from upside down. The parameters for current posture classification formulas should also be modified accordingly.

\subsection{New functions}
The improvement of current system includes five dimensions. The first dimension is to integrate some audio feedback into the feedback system. For example, when current intervention extends of feedback is obtrusive, which means the detected bad posture has already kept for a while, some sound effect can be added to further remind the user about their posture. This would be particularly useful when a feedback for a low viewing height posture is needed in a using mobile device scenario. The action for the user who get used to have a low viewing height posture when using the mobile device to receive a visual feedback for the posture on a mobile device, often make the user to repeat the posture. Therefore, an additional audio feedback could be helpful.

The second dimension is to build an online platform to enable the performance for the posture behaviour of each user team automatically uploaded. The function could hopefully trigger the social influence of social cohesion and conformity for the users on the task of posture improvement. The third dimension is to show the diagram or the picture, which is easily understandable and relevant to the considerations of having good posture for the user, every time when the user start or close the system. It is for improving the attitude of emphasis towards the importance of posture, and the emphasis of having good posture to some extend is a premise for the system to have a long-term effect

The fourth dimensions is to increase the amount of the stages for the Individual Head Tracking to provide from the spot to the skeleton. The spot could change from very transparent to not transparent, and the eyes could appear from very vague to clear. This would make the feedback at the beginning more subtle and reduce the interference to the users.

The fifth dimension is to make a user interface to provide the user with some flexibility to customise the system within some extends. The user could be allowed to set the size and the position of the angel and devil figure, as well as the flower and bomb. They could also adjust the size and colour of the avatar, which are their representatives. The function could not only enlarge the flexibility of the system, but also increase the degree of the favour the users hold to the system.

The implementation for the functions of the four dimension with a refined posture classification model can be expected to enhance the usability of the system especially on the efficiency, flexibility, and user experience aspects, and also increase the effect for the user's' posture improvement. Some new feedback methods could also be added into the system after the evaluation procedure.
