Bergqvist, U., Wolgast, E., Nilsson, B., & Voss, M. (1995). The influence of VDT work on musculoskeletal disorders. Ergonomics, 38(4), 754–762. http://doi.org/10.1080/00140139508925147\\
Biggs, J. (2013, August 21). The LUMOback Is An Infuriating, Ingenious Posture-Saving Device That May Drive You Crazy. Retrieved 8 June 2015, from http://techcrunch.com/2013/08/21/the-lumoback-is-an-infuriating-ingenious-posture-saving-device-that-may-drive-you-crazy/\\
Cialdini, R. B., & Goldstein, N. J. (2004). Social Influence: Compliance and Conformity. Annual Review of Psychology, 55(1), 591–621. http://doi.org/10.1146/annurev.psych.55.090902.142015\\
Clark, R. A., Pua, Y.-H., Fortin, K., Ritchie, C., Webster, K. E., Denehy, L., & Bryant, A. L. (2012). Validity of the Microsoft Kinect for assessment of postural control. Gait & Posture, 36(3), 372–377. http://doi.org/10.1016/j.gaitpost.2012.03.033\\
Dostal, J., Kristensson, P. O., & Quigley, A. (2014). Estimating and using absolute and relative viewing distance in interactive systems. Pervasive and Mobile Computing, 10, 173–186. http://doi.org/10.1016/j.pmcj.2012.06.009\\
Dunne, L. E., Walsh, P., Hermann, S., Smyth, B., & Caulfield, B. (2008). Wearable Monitoring of Seated Spinal Posture. IEEE TRANSACTIONS ON BIOMEDICAL CIRCUITS AND SYSTEMS, 2(2).\\
Dutta, T. (2012). Evaluation of the KinectTM sensor for 3-D kinematic measurement in the workplace. Applied Ergonomics, 43(4), 645–649. http://doi.org/10.1016/j.apergo.2011.09.011\\
Gerr, F., Marcus, M., & Monteilh, C. (2004). Epidemiology of musculoskeletal disorders among computer users: lesson learned from the role of posture and keyboard use. Journal of Electromyography and Kinesiology, 14(1), 25–31. http://doi.org/10.1016/j.jelekin.2003.09.014\\
Green, M. C. (2004). Understanding Media Enjoyment: The Role of Transportation Into Narrative Worlds. Communication Theory, 14(4), 311–327. http://doi.org/10.1093/ct/14.4.311\\
Hargrove, T. (2014, September 1). Does Bad Posture Cause Back Pain? Retrieved 1 June 2015, from http://www.bettermovement.org/2014/does-bad-posture-cause-back-pain/\\
Hogg, M. A. (2000). Social Identity and Social Comparison. Handbook of Social Comparison, 401–421. http://doi.org/10.1007/978-1-4615-4237-7_19\\
How to sit correctly - Live Well - NHS Choices. (2015, April 7). Retrieved 28 May 2015, from http://www.nhs.uk/Livewell/workplacehealth/Pages/howtositcorrectly.aspx\\
IllumiRoom - Microsoft Research. (2015). Retrieved from http://research.microsoft.com/en-us/projects/illumiroom/\\
Improve Your Posture Today With The Lumo Back Posture Sensor. (n.d.). Retrieved 8 June 2015, from http://www.lumobodytech.com/lumoback/\\
Keefer, A. (2013, August 6). Back Problems Caused by Bad Posture While Sitting. Retrieved 1 June 2015, from http://www.livestrong.com/article/346837-back-problems-caused-by-bad-posture-while-sitting/\\
Kinect Sports dev hit by redundancies. (2011, February 16). Retrieved 8 June 2015, from http://www.eurogamer.net/articles/2011-02-16-kinect-sports-dev-hit-by-redundancies\\
LI, G., & BUCKLE, P. (1999). Current techniques for assessing physical exposure to work-related musculoskeletal risks, with emphasis on posture-based methods. Ergonomics, 42(5), 674–695. http://doi.org/10.1080/001401399185388\\
Laeser, K., Maxwell, L., & Hedge, A. (1998). The Effect of Computer Workstation Design on Student Posture. Journal of Research on Computing in Education, 31:2, 173–188.
Lasky, M. S. (2013, March 15). Review: LUMOback Posture Belt. Retrieved 2 June 2015, from http://www.wired.com/2013/03/lumoback/\\
Lethem, J., Slade, P. D., Troup, J. D. G., & Bentley, G. (1983). Outline of a fear-avoidance model of exaggerated pain perception—I. Behaviour Research and Therapy, 21(4), 401–408. http://doi.org/10.1016/0005-7967(83)90009-8\\
Li, H. (2010). The effectiveness of visual prompts on stereotypic behavior ofchildren with autism in elementary schools.\\
MindandMuscle. (2006, November 8). Correcting Posture: Myth or Reality? Retrieved 30 May 2015, from http://www.mindandmuscle.net/articles/correcting-posture-myth-or-reality/\\
Nielson, J. (1995, January 1). 10 Heuristics for User Interface Design: Article by Jakob Nielsen. Retrieved 11 August 2015, from http://www.nngroup.com/articles/ten-usability-heuristics/\\
Output device. (n.d.). [Computer Help]. Retrieved 25 June 2015, from http://www.computerhope.com/jargon/o/outputde.htm\\
Petty, R. E., & Cacioppo, J. T. (1986). The Elaboration Likelihood Model of Persuasion. Communication and Persuasion, 1–24. http://doi.org/10.1007/978-1-4612-4964-1_1\\
Priming. (n.d.). Retrieved 2 June 2015, from https://www.interaction-design.org/encyclopedia/priming.html\\
Riley, M. W., Hovland, C. I., Janis, I. L., & Kelley, H. H. (1954). Communication and Persuasion: Psychological Studies of Opinion Change. American Sociological Review, 19(3). http://doi.org/10.2307/2087772\\
Ronca, D. (n.d.). Is too much TV really bad for your eyes? Retrieved 5 June 2015, from http://health.howstuffworks.com/human-body/systems/eye/tv-bad-for-eyes.htm\\
RoomAliveToolkit, Microsoft Research [Computer Software]. (2015). Retrieved from https://github.com/Kinect/RoomAliveToolkit\\
Scheier, M. F., & Carver, C. S. (1977). Self-focused attention and the experience of emotion: Attraction, repulsion, elation, and depression. Journal of Personality and Social Psychology, 35(9), 625–636. http://doi.org/10.1037//0022-3514.35.9.625\\
Smirnov, A. (2010, July 29). Monitor Support Apparatus.\\
Social Cohesion. (2013). Philosophical Perspectives on Social Cohesion : New Directions for Educational Policy. http://doi.org/10.5040/9781472553171.ch-001\\
Stemland, I., Ingebrigtsen, J., Christiansen, C. S., Jensen, B. R., Hanisch, C., Skotte, J., & Holtermann, A. (2015). Validity of the Acti4 method for detection of physical activity types in free-living settings: comparison with video analysis. Ergonomics, 58(6), 953–965. http://doi.org/10.1080/00140139.2014.998724\\
System Usability Scale (SUS). (2013, September 6). Retrieved 11 August 2015, from http://www.usability.gov/how-to-and-tools/methods/system-usability-scale.html\\
Torres, M. (n.d.). What is the Best Viewing Distance to Watch a TV From? Retrieved 11 June 2015, from http://tv.about.com/od/frequentlyaskedquestions/f/viewingdistance.htm\\
Zhang, S., McCullagh, P., Nugent, C., Zheng, H., & Black, N. (2011). An Ontological Approach for Context-Aware Reminders in Assisted Living’ Behavior Simulation. In Advances in Computational Intelligence (pp. 677–684). Volume 6692.\\
Zhang, S., McCullagh, P., Zhang, J., & Yu, T. (2014). A smartphone based real-time daily activity monitoring system. Cluster Computing, 17(3), 711–721. http://doi.org/10.1007/s10586-013-0335-y\\
Perceptual Constancy | Psychology. (2014, January 6). In Encyclopædia Britannica. Encyclopædia Britannica. Retrieved from http://www.britannica.com/EBchecked/topic/451073/perceptual-constancy