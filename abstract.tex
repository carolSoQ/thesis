\begin{abstract}
\begin{abstract}
The aim of the research is to build an interactive system for posture improvement used in home setting. The target postures are defined from a trade-off between a combination of posture guideline and a predicted relaxing user experience. The posture would be observed by a Kinect. A feedback model that contains 33 feedback methods was developed to enable some appropriate methods can be chosen from for the implementation after evaluation. The feedback methods in the model have 7 targeting posture types, 5 output devices, and 3 intervention extents. A feedback with less intervention extent would be delivered earlier for reducing the interference to the users' current activity.A user study with 12 participants was carried to building a posture classification model, set the baseline of posture behaviour pattern in two scenarios, and investigate the perspectives related to the system design. 

The feedback system include three feedback methods, which are delivered by the projector and the mobile device.The feedback methods can visualise the general posture  behaviour of all the users in the environment, provide ongoing reinforcement, illustrate the impact of current posture individually, and enable active exploration of current posture.The evaluation of the system was done by the comparisons between the posture behaviour measured in the scenarios before and after the system was applied and an interview for investigating the 8 dimensions of usability of the system with 11 participants. The occurrence rate of every target posture was found decreased in every scenario, and the usability in 8 dimensions was rated 4.4 on average on a Likert 5-point scale. The posture classification model can be verified using the data from more people and an UI can be added to improve the flexibility of the system in the future.
\end{abstract}
\end{abstract}